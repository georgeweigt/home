\documentclass[12pt]{article}
\usepackage{parskip}
\begin{document}

{\bf Question 19}

\begin{verbatim}
data work.empsalary
merge work.employee (in = inemp) work.salary (in = insal);
by name;
if inemp and insal;
run;
\end{verbatim}

\bigskip
{\bf Notes}
\begin{enumerate}
\item
The BY statement makes it a match-merge.
\item
The IN= option creates a boolean variable that indicates
whether the data set contributed data to the current observation.
\item
The subsetting IF statement requires both data sets to contribute
to the observation.
\end{enumerate}

\newpage

{\bf Question 28}

The following SAS program is submitted.

\begin{verbatim}
data work.test;
First = 'Ipswich, England';
City = substr(First,1,7);
City_Country = City!!' , '!!�England';
run;
\end{verbatim}

Which one of the following is the value of the variable CITY\_COUNTRY
in the output data set?

\begin{tabular}{ll}
A. & \verb$Ipswich!!$\\
B. & \verb$Ipswich, England$\\
C. & \verb$Ipswich, 'England'$\\
D. & \verb$Ipswich          , England$
\end{tabular}

Answer: D

\bigskip
{\bf Notes}

\begin{enumerate}
\item
The trick is that the compiler gives CITY a length
equal to the length of FIRST since CITY has not been given a length previously.
\item
Recall that SAS does static initialization of character variable lengths.
The SUBSTR function could potentially use a variable for the
length argument instead of a literal.
In that case the compiler would not be able to determine the SUBSTR length
since a variable has no value at compile-time.
Hence for consistency the SUBSTR length argument is not used
to set the length of CITY.
\item
The !!~does string concatenation.
\end{enumerate}

\newpage

{\bf Question 29}

The following SAS program is submitted.

\begin{verbatim}
data numrecords;
infile �file-specification�;
input@1 patient $15.
relative$ 16-26@;
if relative = �children� then
input diagnosis $15. @;
else if relative = �parents� then
input @28 doctor $15.
clinic $ 44-53
@54diagnosis $15. @;
input age;
run;
\end{verbatim}

How many raw data records are read during each iteration of the DATA step during execution?

\begin{tabular}{ll}
A. & 1\\
B. & 2\\
C. & 3\\
D. & 4
\end{tabular}

Answer: A

\bigskip
{\bf Notes}

\begin{enumerate}
\item
The trick is the trailing at-sign.
The trailing at-sign holds the current input record so it can be read again
by another INPUT statement.
\end{enumerate}

\newpage

{\bf Question 30}

The following SAS program is submitted.

\begin{verbatim}
data work.empsalary;
set work.people (in = inemp)
work.money (in = insal);
if insal and inemp;
run;
\end{verbatim}

The SAS data set WORK.PEOPLE has 5 observations,
and the data set WORK.MONEY has 7 observations.

How many observations will the data set WORK.EMPSALARY contain?

\begin{tabular}{ll}
A. & 0\\
B. & 5\\
C. & 7\\
D. & 12
\end{tabular}

Answer: A

\bigskip
{\bf Notes}

\begin{enumerate}
\item
All of the observations are read from the first data set,
then all of the observations are read from the second data set.
There is never a case when both data sets contribute to the
same observation.
Hence the subsetting ``if'' statement is never true.
\item
See Prep Guide p.~367.
\end{enumerate}

\newpage

{\bf Question 31}

The contents of the raw data file SIZE are listed below.

\begin{verbatim}
----|----10---|----20---|----30
72 95
\end{verbatim}

The following SAS program is submitted.

\begin{verbatim}
data test;
infile �size�;
input@1 height 2. @4 weight 2;
run;
\end{verbatim}

Which one of the following is the value of the variable
WEIGHT in the output data set?

\begin{tabular}{ll}
A. & 2\\
B. & 72\\
C. & 95\\
D. & . (missing numeric value)
\end{tabular}

Answer: A

\bigskip
{\bf Notes}

\begin{enumerate}
\item
In the INPUT statement, WEIGHT is followed by 2 which is a column position,
not ``2.''~which would be an informat.
\end{enumerate}

\newpage

{\bf Question 32}

The observations in the SAS data set WORK.TEST are ordered by the values
of the variable SALARY.

The following SAS program is submitted.

\begin{verbatim}
proc sort data = work.test out = work.testsorted;
by name;
run;
\end{verbatim}

Which one of the following is the result of the SAS program?

{\small
\begin{tabular}{ll}
A. & The data set WORK.TEST is stored in ascending order
by values of the NAME variable.\\
B. & The data set WORK.TEST is stored in descending order
by values of the NAME variable\\
C. & The data set WORK.TESTSORTED is stored in ascending order
by values of the NAME variable.\\
D. & The data set WORK.TESTSORTED is stored in descending order
by values of the NAME variable.
\end{tabular}
}

Answer: C

\bigskip
{\bf Notes}

\begin{enumerate}
\item
The default sort order is ascending.
\end{enumerate}

\newpage

{\bf Question 33}

The following SAS program is submitted.

\begin{verbatim}
data allobs;
set sasdata.origin (firstobs = 75 obs = 499);
run;
\end{verbatim}

The SAS data set SASDATA.ORIGIN contains 1000 observations.
How many observations does the ALLOBS data set contain?

\begin{tabular}{ll}
A. & 424\\
B. & 425\\
C. & 499\\
D. & 1000
\end{tabular}

Answer: B

\bigskip
{\bf Notes}

\begin{enumerate}
\item
The number of observations is $499-75+1=425$.
\end{enumerate}

\newpage

{\bf Question 34}

A raw data file is listed below.

\begin{verbatim}
John McCloskey 35 71
June Rosesette 1043
Tineke Jones 9 37
\end{verbatim}

The following SAS program is submitted using the raw data file as input.

\begin{verbatim}
data work.homework;
infile 'file-specification';
input name $ age height
if age LE 10;
run;
\end{verbatim}

How many observations will the WORK.HOMEWORK data set contain?

\begin{tabular}{ll}
A. & 0\\
B. & 2\\
C. & 3\\
D. & No data set is created as the program fails to execute due to errors.
\end{tabular}

Answer: C

\bigskip
{\bf Notes}

\begin{enumerate}
\item
What happens is the NAME variable only scans the first name.
Then the AGE variable gets set to ``missing'' because the scanner is
on the last name.
According to SAS documentation,
``A missing numeric value is smaller than any other numeric value.''
Hence the subsetting if-statement is always true.
\end{enumerate}

\newpage

{\bf Question 37}

The following SAS program is submitted.

\begin{verbatim}
data _null_;
set old;
put sales1 sales2;
run;
\end{verbatim}

Where is the output written?

\begin{tabular}{ll}
A. & the SAS log\\
B. & the raw data that was opened last\\
C. & the SAS output window or an output file\\
D. & the data set mentioned in the DATA statement
\end{tabular}

Answer: A

\bigskip
{\bf Notes}

\begin{enumerate}
\item
The DATA \verb$_$NULL\verb$_$ statement does not create a data set,
but the PUT statement still works.
Recall that PUT writes to the SAS log.
\end{enumerate}

\newpage

{\bf Question 38}

When the following SAS program is submitted,
the data set SASDATA.PRDSALES contains 5000
observations.

\begin{verbatim}
libname sasdata 'SAS-data-Iibrary';
options obs = 500;
proc print data = sasdata.prdsales (firstobs = 100);
run;
options obs = max;
proc means data = sasdata.prdsales (firstobs = 500);
run;
\end{verbatim}

How many observations are processed by each procedure?

\begin{tabular}{ll}
A. & 400 for PROC PRINT\\
& 4500 for PROC MEANS\\
B. & 401 for PROC PRINT\\
& 4501 for PROC MEANS\\
C. & 401 for PROC PRINT\\
& 4500 for PROC MEANS\\
D. & 500 for PROC PRINT\\
& 5000 for PROC MEANS
\end{tabular}

Answer: B

\bigskip
{\bf Notes}

\begin{enumerate}
\item
This program demonstrates that settings such as OBS and FIRSTOBS can
be set two ways.
\item
This program also demonstrates that MAX basically removes the OBS limit.
\item
Recall that OBS is the last observation to use and that
the total number of observations is
$$\rm OBS-FIRSTOBS+1$$
\end{enumerate}

\end{document}