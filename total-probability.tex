\documentclass[12pt]{article}
\usepackage{amsmath,amsfonts,mathrsfs,graphicx,parskip,fancyvrb}
\begin{document}

From {\it The Week.}
\begin{quote}
Suppose one out of every million people is a terrorist
(if anything, an overestimate), and you've got a machine
that can determine whether someone is a terrorist with 99.9 percent accuracy.
You've used the machine on Mr.~X, and it gives a positive result.
What are the odds that Mr.~X is a terrorist?
Here's the answer: a 0.1 percent chance --- which is to say,
the 99.9 percent accurate test will give you the wrong answer 99.9
percent of the time.\footnote{
Cooper, Ryan.
{\it The simple math problem that blows apart the NSA's surveillance justifications.}
http://theweek.com/articles/547119/simple-math-problem-that-blows-apart-nsas-surveillance-justifications}
\end{quote}

Let $X=1$ be the event that Mr.~X is a terrorist with probability
one in a million.
\begin{align*}
P(X{=}1)&=0.000001\\
P(X{=}0)&=0.999999
\end{align*}
Let $M=1$ be the event that the machine gives a positive result.
The accuracy of the machine is a conditional probability.
Given the condition that Mr.~X is a terrorist ($X=1$),
the probability that the machine gives a positive result ($M=1$)
is 99.9 percent.
\[
P(M{=}1\mid X{=}1)=0.999
\]
The trick is to swap $M$ and $X$.
In other words, determine the conditional probability of $X=1$ given that $M=1$.
A conditional probability can be ``turned around'' using the following formula.
\[
P(X{=}1\mid M{=1})=\frac{P(M{=}1\mid X{=}1)P(X{=}1)}{P(M{=}1)}
\]
The denominator $P(M{=}1)$ can be obtained from the law of total probability.
\[
P(M{=}1)=P(M{=}1\mid X{=}1)P(X{=}1)+P(M{=}1\mid X{=}0)P(X{=}0)
\]
The term $P(M{=}1\mid X{=}0)$ is the probability of a false positive.
Since the machine is 99.9 percent accurate,
the probability the machine is wrong is 0.1 percent.
\[
P(M{=}1\mid X{=}0)=0.001
\]
Hence by the law of total probability
\begin{align*}
P(M{=}1)
&=P(M{=}1\mid X{=}1)P(X{=}1)+P(M{=}1\mid X{=}0)P(X{=}0)\\
&=0.999\times0.000001+0.001\times0.999999\\
&=0.001001
\end{align*}
The probability that the machine is correct when it signals a positive result is
\begin{align*}
P(X{=}1\mid M{=1})
&=\frac{P(M{=}1\mid X{=}1)P(X{=}1)}{P(M{=}1)}\\
&=\frac{0.999\times0.000001}{0.001001}\\
&=0.000998
\end{align*}
So the probability is indeed about 0.1 percent.

\newpage

The following hypothetical problem is from an MIT online course in Economics.
MIT is planning to subject admitted freshman to
``At Home Test Kit for Illicit Drugs'' which has been shown to be fairly
reliable, in the sense that 90\% of those using drugs test positive,
while only 10\% of those not using drugs test positive.
Assume that 20\% of the population in your age group actually uses illicit drugs.
Alas, you test positive.
What is the probability that you are truly and fairly busted?

Let $X$ be the event that the test is positive.
Let $U$ be the event that the student uses drugs.
The following are given.
\begin{align*}
P(X\mid U)&=0.9\\
P(X\mid \bar U)&=0.1\\
P(U)&=0.2
\end{align*}
This is what is sought:
The conditional probability that the student
uses drugs given that the test result is positive.
\[
P(U\mid X)=\frac{P(X\cap U)}{P(X)}
\]
The value of $P(X\cap U)$ is obtained from the definition of
conditional probability.
\[
P(X\cap U)=P(X\mid U) P(U)=0.9\times0.2=0.18
\]
The value of $P(X)$ is obtained from total probability.
\[
P(X)=P(X\mid U)P(U)+P(X\mid\bar U)P(\bar U)
=0.9\times0.2+0.1\times0.8=0.26
\]
Hence
\[
P(U\mid X)=\frac{P(X\cap U)}{P(X)}=\frac{0.18}{0.26}=0.6923
\]

\newpage

\section*{MIT Problem}

\begin{center}
\begin{tabular}{lrrr}
\hline
& Sample Size & Number & Percentage\\
Group & Enrolled & Placed in Jobs & Placed in Jobs\\
\hline
Tax Credit Voucher & 247 & 32 & 13.0\\
Direct Rebate Voucher & 299 & 38 & 12.7\\
Control & 262 & 54 & 20.6\\
Total & 808 & 124 & 15.3\\
\hline
\end{tabular}
\end{center}

1. Construct separate $t$-tests comparing each of the two
treatment groups with the control group.
Did the treatments have statistically significant effects? {\it Yes.}
If so, in what direction? {\it The wrong direction.}

\begin{Verbatim}[fontsize=\small]
> TaxCredit = numeric(247)
> TaxCredit[1:32] = 1
> DirectRebate = numeric(299)
> DirectRebate[1:38] = 1
> Control = numeric(262)
> Control[1:54] = 1
> t.test(TaxCredit,Control,var.equal=TRUE)

	Two Sample t-test

data:  TaxCredit and Control
t = -2.3111, df = 507, p-value = 0.02123
alternative hypothesis: true difference in means is not equal to 0
95 percent confidence interval:
 -0.14162910 -0.01147533
sample estimates:
mean of x mean of y 
0.1295547 0.2061069 

> t.test(DirectRebate,Control,var.equal=TRUE)

	Two Sample t-test

data:  DirectRebate and Control
t = -2.5317, df = 559, p-value = 0.01163
alternative hypothesis: true difference in means is not equal to 0
95 percent confidence interval:
 -0.1403224 -0.0177107
sample estimates:
mean of x mean of y 
0.1270903 0.2061069 
\end{Verbatim}

2. Test whether the employment rates in the two treatment groups differ
from each other.
{\it No statistical difference.}

\begin{Verbatim}[fontsize=\small]
> t.test(TaxCredit,DirectRebate,var.equal=TRUE)

	Two Sample t-test

data:  TaxCredit and DirectRebate
t = 0.0856, df = 544, p-value = 0.9318
alternative hypothesis: true difference in means is not equal to 0
95 percent confidence interval:
 -0.05410503  0.05903374
sample estimates:
mean of x mean of y 
0.1295547 0.1270903 
\end{Verbatim}

3. Construct 95\% confidence intervals for the two treatment-control contrasts.

\begin{Verbatim}[fontsize=\small]
data:  TaxCredit and Control
95 percent confidence interval:
 -0.14162910 -0.01147533

data:  DirectRebate and Control
95 percent confidence interval:
 -0.1403224 -0.0177107
\end{Verbatim}

4. Footnote 11 reports a chi-square statistic for independence between groups
(tax credit, rebate, and control) and job placement.
Explain how this was constructed.

\end{document}
