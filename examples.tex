\documentclass[12pt]{article}
\usepackage[margin=2cm]{geometry}
\usepackage{amsmath}
\usepackage{slashed}
\usepackage{tikz}
\usepackage{verbatim}

\usepackage{hyperref}
\hypersetup{colorlinks=true,urlcolor=blue}

\begin{document}

\noindent
Example 1. Compute the probability $p$ that 23 people have different birthdays.
$$
p=\frac{365}{365}\times\frac{364}{365}\times\frac{363}{365}\times\cdots
\times\frac{343}{365}=\frac{365!/(365-23)!}{365^{23}}
$$

\noindent
The probability that at least two people have the same birthday is $1-p$.

\begin{verbatim}
"Product method"
p = product(k,1,23,(365-k+1)/365)
float(p)
"Factorial method"
p = 365! / (365 - 23)! / 365^23
float(p)
"Probability of at least one shared birthday"
1.0 - p
\end{verbatim}

\bigskip
\hrule

\bigskip
\noindent
Example 2. Show that
$$
\frac{E^2+m^2+p^2\cos\theta}{8p^4}
=\frac{1-\beta^2\sin^2\theta/2}{4p^2\beta^2}
$$
where $p=\sqrt{E^2-m^2}$ and $\beta=p/E$.

\begin{verbatim}
p = sqrt(E^2 - m^2)
beta = p/E
A = (E^2 + m^2 + p^2 cos(theta)) / (8 p^4)
B = (1 - beta^2 sin(theta/2)^2) / (4 p^2 beta^2)
A == B
\end{verbatim}

\bigskip
\hrule

\bigskip
\noindent
Example 3. Let
$$
{\bf E}=
\begin{pmatrix}
A\sin(kz-\omega t+\phi)\\0\\0
\end{pmatrix}
\qquad\text{and}\qquad
{\bf B}=
\begin{pmatrix}
0\\A\sin(kz-\omega t+\phi)\\0
\end{pmatrix}
$$
where $k=\omega/c$.
Verify that $\bf E$ and $\bf B$ are solutions to the free-field Maxwell equations
\begin{gather*}
\nabla\cdot{\bf E}=0\\
\nabla\cdot{\bf B}=0\\
\nabla\times{\bf E}+\frac{1}{c}\frac{\partial}{\partial t}{\bf B}=\begin{pmatrix}0\\0\\0\end{pmatrix}\\
\nabla\times{\bf B}-\frac{1}{c}\frac{\partial}{\partial t}{\bf E}=\begin{pmatrix}0\\0\\0\end{pmatrix}
\end{gather*}

\begin{verbatim}
k = omega/c
E = (A sin(k z - omega t + phi), 0, 0)
B = (0, A sin(k z - omega t + phi), 0)
div(E) == 0
div(B) == 0
curl(E) + d(B,t)/c == 0
curl(B) - d(E,t)/c == 0
\end{verbatim}

\bigskip
\hrule

\bigskip
\noindent
Example 4. Let
$$
\psi=\exp(ik_xx+ik_yy+ik_zz-i\omega t)
$$

\noindent
where
$$
\omega=\sqrt{k_x^2+k_y^2+k_z^2+m^2}
$$

\noindent
Verify that $\psi$ is a solution to the Klein-Gordon equation
$$
\frac{\partial^2}{\partial t^2}\psi
-\frac{\partial^2}{\partial x^2}\psi
-\frac{\partial^2}{\partial y^2}\psi
-\frac{\partial^2}{\partial z^2}\psi
+m^2\psi=0
$$

\begin{verbatim}
omega = sqrt(kx^2 + ky^2 + kz^2 + m^2)
psi = exp(i kx x + i ky y + i kz z - i omega t)
d(psi,t,t) - d(psi,x,x) - d(psi,y,y) - d(psi,z,z) + m^2 psi == 0
\end{verbatim}

\bigskip
\hrule

\bigskip
\noindent
Example 5. Show that
$$
u_1\bar{u}_1+u_2\bar{u}_2=(E+m)(\slashed{p}+m)
$$

\noindent
where
$$
E=\sqrt{p_x^2+p_y^2+p_z^2+m^2}
$$

\noindent
The spinors are
$$
u_1=\begin{pmatrix}E+m\\0\\p_z\\p_x+ip_y\end{pmatrix}
\qquad
u_2=\begin{pmatrix}0\\E+m\\p_x-ip_y\\-p_z\end{pmatrix}
$$

\noindent
The adjoint of $u$ is $\bar{u}=u^\dag\gamma^0$.
Note that $\bar{u}$ is a row vector hence $u\bar{u}$ is an outer product.

\begin{verbatim}
E = sqrt(px^2 + py^2 + pz^2 + m^2)
u1 = (E + m, 0, pz, px + i py)
u2 = (0, E + m, px - i py, -pz)
p = (E, px, py, pz)
I = ((1,0,0,0),(0,1,0,0),(0,0,1,0),(0,0,0,1))
gmunu = ((1,0,0,0),(0,-1,0,0),(0,0,-1,0),(0,0,0,-1))
gamma0 = ((1,0,0,0),(0,1,0,0),(0,0,-1,0),(0,0,0,-1))
gamma1 = ((0,0,0,1),(0,0,1,0),(0,-1,0,0),(-1,0,0,0))
gamma2 = ((0,0,0,-i),(0,0,i,0),(0,i,0,0),(-i,0,0,0))
gamma3 = ((0,0,1,0),(0,0,0,-1),(-1,0,0,0),(0,1,0,0))
gamma = (gamma0,gamma1,gamma2,gamma3)
pslash = dot(gmunu,p,gamma)
bar(u) = dot(conj(u),gamma0)
A = outer(u1,bar(u1)) + outer(u2,bar(u2))
B = (E + m) (pslash + m I)
A == B
\end{verbatim}

\end{document}
